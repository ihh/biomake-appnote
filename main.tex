\documentclass{bioinfo}

\usepackage{fancyvrb}
\usepackage{float}

\copyrightyear{2016} \pubyear{2016}

\access{Advance Access Publication Date: Day Month Year}
\appnotes{Application Note}

\usepackage{url}

% Definitions
\newcommand\Makefile{{\tt Makefile}}
\newcommand\make{GNU Make}
\newcommand\biomake{BioMake}
\newcommand\footurl[1]{\footnote{\url{#1}, accessed Dec 9, 2016.}}

% Document
\begin{document}
\firstpage{1}

\subtitle{Application Note}

\title[BioMake: declarative workflow management]{BioMake: a GNU Make-compatible utility for declarative workflow management}
\author[Ian Holmes and Chris Mungall]{Ian Holmes$^{\text{\sfb 1,2}*}$ and Chris Mungall$^{\text{\sfb 2}*}$}
\address{$^{\text{\sf 1}}$Department of Bioengineering, University of California, Berkeley, CA 94720, USA and \\
$^{\text{\sf 2}}$Lawrence Berkeley National Laboratory, 1 Cyclotron Rd, Berkeley, CA 94720, USA.}

\corresp{$^\ast$To whom correspondence should be addressed.}

\history{}

\editor{}

\abstract{
{\bf Motivation.}
The Unix ``make'' program is widely used in bioinformatics pipelines,
but suffers from problems that limit its application to large analysis datasets.
These include reliance on file modification times to determine whether a target is stale,
lack of support for parallel execution on clusters, and restricted flexibility to extend the underlying logic program.
{\bf Results.}
We present BioMake, a make-like utility that is compatible with most features of GNU Make
and adds support for popular cluster-based job-queue engines, MD5 signatures as an alternative to timestamps, and logic programming extensions in Prolog.
{\bf Availability and Implementation.}
BioMake is available from \url{https://github.com/cmungall/biomake} under the Creative Commons Attribution 3.0 US license.
The only dependency is SWI-Prolog, available from \url{http://www.swi-prolog.org/}.
{\bf Contact.}
Ian Holmes {\tt ihholmes+biomake@gmail.com} or Chris Mungall {\tt cmungall+biomake@gmail.com}.
{\bf Supplementary Information.}
None.
}

\maketitle

\section*{Introduction}

The familiar Unix \make\ utility has become a favored tool for ``bioinformatics in-the-large'' \citep{ParkerGorlickLee2003}.
Alongside many more elaborate workflow management systems, \make\ holds its own for several reasons.
Besides being ubiquitous and easy to use, with a simple syntax, it offers a powerful mix
of {\em declarative logic} (the specification of target-dependency relationships at the head of \Makefile\ recipes, from which \make\ deduces build chains)
with {\em Unix scripting} (the lines of shell script that make up the body of recipes).
GNU Make combines these elements with {\em functional programming}-inspired manipulation of text variables, lists, and directories
(GNU Make offers a number of built-in functions for such purposes, and includes Guile --- GNU's Scheme interpreter --- as an extension language).

In our usage of \make\ for data analysis, a common pattern is to analyze one or two examples manually, building up a \Makefile\ recipe (or set of recipes),
then to scale the analysis up to the whole dataset.
Makefiles remain, in our opinion, unrivalled for this purpose.
However, GNU Make's origins were as a tool for managing build pipelines, not large-scale data analyses, and it has several flaws that impede its use in bioinformatics.

\section*{Results}

We have developed a new tool, \biomake, that keeps the best features of \make\ (including the ability to read a GNU \Makefile) while addressing its shortcomings.
Chief innovations of \biomake\ include:

{\bf (1) MD5 signatures as an alternative to time-stamps.}
\make\ uses file modification times to determine when files need to be rebuilt.
This is fragile, especially on networked filesystems or cloud storage, where file timestamps may not be preserved or synchronized.
In projects where a big data analysis can take hours or days, a spurious rebuild can be devastating, especially if it triggers further rebuilding of downstream targets.
Instead of using timestamps, \biomake\ can be directed to use MD5 checksums: whenever a target is built, the MD5 hashes of that file and its dependents are recorded
and stored. This can be used in combination with \Makefile\ recipes that sort or canonicalize data to further prevent spurious rebuilds.
For example, in the \Makefile\ snippet shown in Figure~1,
if {\tt sortfasta} is a utility that sorts sequences in a FASTA file and regularizes them to 50 characters per line, then changing the sequence order or formatting
of a file {\tt seqs.fasta} will not cause the BLAST job for {\tt seqs.blast} to be re-run.

\begin{figure}[H]
\begin{Verbatim}[frame=single]
%.sorted.fasta: %.fasta
    sortfasta $< >$@

%.blast: %.sorted.fasta
    blastn -db $(DATABASE) -query $< >$@
\end{Verbatim}
\caption{
  A hypothetical \Makefile\ that sorts a sequence file before BLASTing it.
  If MD5 signatures (rather than timestamps) are used to trigger builds, as is an option in \biomake,
  then a reformatting of the {\tt .fasta} file that leads to an identical {\tt .sorted.fasta} file
  will not trigger a rebuild of the {\tt .blast} file, even if the timestamps have changed.
}
\end{figure}

{\bf (2) Support for cluster-based job queues.}
\make\ can run multiple jobs in parallel, but only on one machine.
It is possible to write cluster support directly into the \Makefile,
wrapping each recipe with a call to a job submission script,
but this spoils \make's otherwise clean separation of concerns
and often prevents it from tracking dependencies properly.
\biomake\ has built-in support for Sun Grid Engine, PBS, and SLURM job submission systems,
including dependency tracking (ensuring a target is not built until all its dependents have been built).
It also (like GNU Make) offers built-in parallel execution on
the same machine that \biomake\ is being run on.

{\bf (3) Multiple wildcards per filename.}
\make\ only allows a single wildcard (``stems'') in a filename,
represented by the percent symbol ({\tt \%}) in the head of a recipe and by the automatic variable {\tt \$*} in the body.
In contrast, \biomake\ allows multiple wildcards: any unbound variable that appears in the head of a recipe can serve as a wildcard,
and can subsequently be used in the body of the recipe.
For example, the recipe in Figure~2 creates an alignment file {\tt \$X.\$Y.alignment} from any two files {\tt \$X.fasta} and {\tt \$Y.fasta}.


\begin{figure}[H]
\begin{Verbatim}[frame=single]
$X.$Y.alignment: $X.fasta $Y.fasta
    align $X $Y >$@
\end{Verbatim}
\caption{
  A hypothetical \Makefile\ that aligns two sequence files.
  In \biomake, if the variables {\tt \$X} and {\tt \$Y} are unbound, then they will be treated as filesystem wildcards
  for the purpose of matching this rule.
}
\end{figure}

{\bf (4) Easy integration with ontologies and description logics.}
GNU Make's domain-specific language extensions are based on Scheme, which is a functional language,
but the underlying structure of a Makefile (rules such as ``{\em to build A, you must first build B}'' and ``{\em to build B, you must first build C and D}'')
is a logic program.
\biomake's domain-specific language is Prolog, making it trivially easy to incorporate ontologies and description logics
such as the Gene Ontology \citep{GeneOntology2000,GeneOntology2015}, the Sequence Feature Ontology \citep{SequenceOntology2005} or the Protein Feature Ontology \citep{ProteinFeatureOntology2008}.
For example, we can easily create \biomake\ recipes for targets such as ``{\em the whole-genome alignment for species X and Y, where X is a mammal and Y is a vertebrate}''
or ``{\em the GFF file containing co-ordinates of every human genomic feature of type T, where T is a term descended from `biological-region' in the Sequence Ontology}''.
In a Scheme program, we would have to write, test, and debug functions that explicitly generated these lists;
in a Prolog database, logical conditions such as ``{\em X is a mammal}'' or ``{\em T is descended from biological-region}'' are trivially easy to model directly,
and the Prolog interpreter itself searches for all variable bindings that fit the model.
An example of this sort of construct is given in the main {\tt README.md} file of the GitHub repository.

% In order to use the Prolog extension language, the \Makefile\ must be translated from GNU Make syntax to a similar Prolog syntax that we call a ``Makeprog''.
% This mapping is straightforward (and can be performed automatically by \biomake): the translation of Figure~2 is shown in Figure~3.
% Following this step, Prolog facts can be inserted directly into the Makeprog; furthermore, Prolog terms can be attached as extra qualifiers to the recipes (Figure~4).
% \begin{figure}[H]
% \begin{Verbatim}[frame=single]
% '$X.$Y.alignment' <-- ['$X.fasta','$Y.fasta'],
%     'align $X $Y >$@'.
% \end{Verbatim}
% \caption{
%   A ``Makeprog'' Prolog translation of the \Makefile\ recipe in Figure~2.
% }
% \end{figure}
% 
% \begin{figure}[H]
% \begin{Verbatim}[frame=single]
% vertebrate(chicken).
% mammal(pig).
% mammal(cow).
% primate(chimp).
% primate(human).
% mammal(X) :- primate(X).
% vertebrate(X) :- mammal(X).
% 
% mammal_pair(X,Y) :-
%  mammal(X),
%  mammal(Y),
%  X @< Y.
% 
% mammals <-- Deps, 
%  {findall( t([X,'.',Y,'.alignment']),
%            mammal_pair(X,Y),
%            Deps )}.
% \end{Verbatim}
% \caption{
%   An addition to the Prolog Makeprog of Figure~3 that includes a target, ``mammals'', to build alignments
%   for all ordered pairs of mammals known to the database (here {\tt @<} is the syntax for alphabetic comparison of Prolog atoms).
% }
% \end{figure}

\section*{Discussion}

We can contrast BioMake with other solutions for bioinformatics workflow management.
Systems such as CWL\footurl{http://commonwl.org/} or Galaxy\footurl{https://usegalaxy.org/} have many useful features such as web interfaces and cloud support,
but they do not deduce the workflow from a series of declaratively-specified dependencies: rather, they require explicit specification of inputs, outputs, and ordering of tasks.

There are also other GNU Make clones and variants, including some in functional/parallel languages (Erlang make\footurl{http://erlang.org/doc/man/make.html}),
offering MD5 signatures (omake\footurl{http://omake.metaprl.org/}, makepp\footurl{http://makepp.sourceforge.net/})
or supporting cluster-based parallelism (e.g. Oracle Grid Engine's qmake\footurl{http://gridscheduler.sourceforge.net/htmlman/htmlman1/qmake.html}).
Some of these overlap in functionality with BioMake, but none offers the full feature set described here.


%\section*{Acknowledgements}
%
%\biomake\ was written by Chris Mungall and Ian Holmes.

\section*{Funding}

This work has been supported by NHGRI grant R01-HG004483.

\bibliographystyle{natbib}
%\bibliographystyle{bioinformatics}
%\bibliographystyle{achemnat}
%\bibliographystyle{plainnat}
%\bibliographystyle{abbrv}
%
%\bibliographystyle{plain}
%
%\bibliography{Document}


\bibliography{references}
\end{document}
